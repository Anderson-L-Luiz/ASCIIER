\documentclass{article}
\usepackage[utf8]{inputenc}
\usepackage{parskip}
\usepackage{hyperref}
\usepackage[T1]{fontenc}
\usepackage{lmodern}
\usepackage{amsmath}
\usepackage{xcolor}         % For colored text
\usepackage{courier}        % For pcr font family
\usepackage{xurl}           % For better URL handling

\title{Commit Summary Report (CUM Report)}
\author{Generated by Git Post-Commit Hook}
\date{\today}

\begin{document}
\maketitle
\begin{abstract}
This document contains summaries of commits, generated automatically after each commit. Each commit is a section, and each modified file within that commit is presented as a subsection with its AI-generated summary. Commit details are provided at the end of each section.
\end{abstract}
\tableofcontents
\newpage

\section{Introduction}
This document provides a log of commit summaries. Each main section corresponds to a single Git commit. Within each commit section, individual subsections detail the AI-generated summary for each modified file. Key details about the commit (hash, author, date) are listed at the end of each section's content.
\section{Commit d9464b5 by Anderson-L-Luiz (2025-05-14 19:01:21 +0000)}
\subsection{File: \texttt{CUM\_report/commit\_log.tex}}\n{\fontfamily{pcr}\selectfont\n The commit `d9464b5` updates the `CUM\_report/commit\_log.tex` file, likely to include summaries of commits after each commit to the repository. The auto--generated document includes commit hashes, authors, dates, and brief summaries of each commit, generated using a model analyzing the diff of the commit. The purpose of this change is to provide an organized log of important commits for the project with descriptions of the changes made in each commit.\n}\n\n\subsection{File: \texttt{install\_cum.sh}}\n{\fontfamily{pcr}\selectfont\n The commit d9464b5 in the file install\_cum.sh introduces changes to the installation and setup of the CUM Report, likely used when a commit is made to a Git repository. 

The script initially checks whether the user is running the script from within a Git repository. It then identifies the root directory of the Git repository and sets a variable GIT\_ROOT. This script then proceeds to install several packages via the command line ('install\_package' function). These packages include 'jq', 'curl', and 'texlive--latex--extra'.

Next, the script will create a directory for the CUM Report if it doesn't already exist. Finally, it creates an initial LaTeX file.

From the changes made, it seems that this script is designed to automate the creation of a commit summary report, which seems to include an AI--generated summary of each modified file within a commit. This report can be generated and updated whenever a commit is made to the repository. The report format is a LaTeX document, with sections and subsections, and follows a similar structure to a standard LaTeX document. It is worth noting that the use of AI--generated summaries could relate to AI tools, like machine learning or natural language processing, integrated with the Git repository, allowing for automated creation of these reports.\n}\n\n\subsection{File: \texttt{optiviz.html}}\n{\fontfamily{pcr}\selectfont\n This commit consists of the HTML and some of the CSS of a web page that is part of a three.js terrain simulation. The functionality of the page was not changed. It focuses mainly on styling changes and adding a new control group for users to interact with.

1. **Introduced new controls**: A new control group has been added to the existing controls, "Terrain Descent Controls", it appears to be form for user inputs and selection, this might be to fine tune the terrain simulation parameters for a particular use case. 

2. **Updated Styling**: There're a few updates to the styling for various input controls, text--based inputs have more padding and may have adjusted width.

3. **Added another checkbox group**: There's another check box group, which we assume for a more specific use case control, and as per the changes its styling seems more defined with more level of detail in the styling.

4. **Some part of the code was truncated so there might be more general styling applied to the whole page.**

The changes are likely to improve the user interaction with the webpage by providing more precise options for controlling the terrain simulation, which in turn would provide a more customized and satisfying user experience. 

Note: This is a hypothetical interpretation, because partial diffs and without context, it's hard to pinpoint the exact changes. The actual changes could possibly have much broader impact.\n}\n\n
{\color{blue}\small % Start color blue and make text small\nCommit: \texttt{d9464b5e5691eb98edfd371e3f37c421b9fcd129} \\\nAuthor: Anderson-L-Luiz \\\nDate: 2025-05-14 19:01:21 +0000\n} % End color blue\n
\hrulefill

\end{document}
